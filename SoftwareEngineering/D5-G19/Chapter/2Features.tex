\chapter{Features}
\begin{table}[ht]
    \centering
    \begin{tabular}{|m{0.09\linewidth}|m{0.2\linewidth}|m{0.63\linewidth}|}
    \hline
        \textbf{User Story} & \textbf{Feature index} & \textbf{Feature} \\
        \hline
        \rowcolor{Gray}
        US1 & F1 & Mostrare i parchi selezionabili utilizzando una lista.\\
        \hline
        US1 & F2 & Estrarre tutti i parchi memorizzati tramite un’API locale.\\
        \hline
        \rowcolor{Gray}
        US1 & F3 & Mostrare la pagina del parco selezionato.\\
        \hline
        US2 & F4 & Mostrare le specie presenti nel parco selezionato utilizzando un elenco.\\
        \hline
        \rowcolor{Gray}
        US2 & F5 & Estrarre tutte le specie presenti nel parco tramite un’API locale.\\
        \hline
        US3 & F6 & Mostra pagina contenente le statistiche ambientali.\\
        \hline
        \rowcolor{Gray}
        US3 & F7 & Estrarre le statistiche ambientali memorizzate tramite un’API locale.\\
        \hline
        US4 & F8 & Mostrare le specie presenti nel parco selezionato utilizzando un elenco.\\
        \hline 
        \rowcolor{Gray}
        US4 & F9 & Estrarre tutte le specie presenti nel parco tramite un’API locale.\\
        \hline
        US4 & F10 & Mostrare pagina con la mappa contenente le posizioni degli animali e l’istogramma sullo storico della popolazione.\\
        \hline 
        \rowcolor{Gray}
        US4 & F11 & Estrarre tutte le posizioni dell’animale selezionato tramite un’API locale.\\
        \hline
        US4 & F12 & Estrarre lo storico dell’animale selezionato tramite un’API locale.\\
        \hline 
        \rowcolor{Gray}
        US5 & F13 & Mostrare pagina con i dati dei sensori GPS tramite un elenco.\\
        \hline
        US5 & F14 & Estrarre i dati dei sensori tramite un’API locale filtrati con il parco selezionato dall’amministratore.\\
        \hline
        \rowcolor{Gray}
        US5 & F15 & Mostrare un’area contenente i dati del sensore selezionato.\\
        \hline
        US5 & F16 & Predisporre un bottone per eliminare il sensore.\\
        \hline
        \rowcolor{Gray}
        US5 & F17 & Aggiornare il database tramite un’API locale.\\
        \hline
        US6 & F18 & Mostrare una pagina con i dati dei sensori tramite un elenco.\\
        \hline
        \rowcolor{Gray}
        US6 & F19 & Estrarre i dati dei sensori tramite un’API locale, filtrati con il parco selezionato dall’amministratore.\\
        \hline
        US6 & F20 & Predisporre un bottone per aggiungere un sensore.\\
        \hline
        \rowcolor{Gray}
        US6 & F21 & Visualizzare un form con dei campi da compilare per aggiungere un sensore.\\
        \hline 
        US6 & F22 & Memorizzare sul database i dati del nuovo sensore tramite un’API locale.\\
        \hline
        \rowcolor{Gray}
        US7 & F23 & Mostrare una pagina contenente i dati dei sensori tramite un elenco.\\
        \hline
        US7 & F24 & Estrarre i dati dei sensori tramite un’API locale filtrati con il parco selezionato dall’amministratore.\\
        \hline 
        \rowcolor{Gray}
        US7 & F25 & Visualizzare un’area contenente i dati del singolo sensore.\\
        \hline
    \end{tabular}
\end{table}