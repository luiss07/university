\section{Back End}
\textbf{Google Maps}
\begin{itemize}
    \item La nostra applicazione implementa \textbf{Google Maps} in quando ci permette di utilizzare le loro mappe che sono sempre in aggiornamento. Inoltre, possiamo mostrare i parcheggi disponibili ed utilizzare la sua funzione di navigazione per fornire il miglior percorso per raggiungere il parcheggio selezionato.
\end{itemize}
\textbf{Autenticazione Google - Apple - Facebook}
\begin{itemize}
    \item Permettiamo ai nostri utenti di effettuare il login attraverso google, facebook ed apple perchè sono un’\textbf{alternativa sicura} per effettuare il login nella nostra applicazione.  
\end{itemize}
\textbf{Videocamera - Sbarre del parcheggio}
\begin{itemize}
    \item La \textbf{videocamera} del parcheggio viene utilizzata per riconoscere le targhe delle macchine che accedono al parcheggio dopo aver prenotato un posto attraverso la nostra applicazione. Così facendo l’accesso può essere effettuato dall’utente facendo alzare le sbarre del parcheggio \textbf{senza dover prendere il biglietto} e registrare l’accesso nel sistema del parcheggio.
\end{itemize}
\textbf{Localizzazione GPS del telefono}
\begin{itemize}
    \item Utilizzata da Google Maps per generare gli itinerari e trovare la posizione del dispositivo.
\end{itemize}
\textbf{Database esterno per i dati degli utenti}
\begin{itemize}
    \item Esterno supportato da \textbf{Amazon Relational Database Service (RDS)}, dove verranno inseriti tutti i dati dell’utente(nome, cognome, codice fiscale, email, numeri di telefono, delle carte e delle targhe). Essi dovranno essere protetti dai software e \textbf{sistemi di criptazione} offerti da RDS.
\end{itemize}
\textbf{Infrastruttura del parcheggio}
\begin{itemize}
    \item L'accesso al sistema del parcheggio è necessario per poter \textbf{aggiornare dinamicamente il numero di posti disponibili}, utilizzando le videocamere per leggere \textcolor{red}{e verificare} le targhe delle macchine in ingresso.
\end{itemize}
\textcolor{red}{\textbf{Google calendar}}
\begin{itemize}
    \item \textcolor{red}{Il sistema utilizza il calendario Google per permettere all'utente di prenotare per un determinato giorno e una determinata ora (nel solo caso in cui la prenotazione sia per un tempo maggiore di 24h)}
\end{itemize}