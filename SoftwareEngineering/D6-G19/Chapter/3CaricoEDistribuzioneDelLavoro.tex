\section{Carico e distribuzione del lavoro}

Dall’analisi dei file di log risulta il seguente carico di lavoro espresso in minuti/persona per ciascun membro del gruppo.
Dalla tabella sottostante si può notare che i principali squilibri sono stati nel D1, nel D5 e nel D6. \\
Nel primo caso la differenza è dovuta soprattutto al fatto che il lavoro non era stato organizzato con un certo criterio, perciò il risultato è stato raggiunto spendendo più ore del necessario e dividendo il carico del lavoro in modo sbagliato.\\
Nel secondo caso, essendo l'obiettivo molto più complesso e data la nostra conoscenza preliminare dell'argomento molto scarsa per non dire nulla, la suddivisione del lavoro è avvenuta sia in base a cosa ciascuno di noi si sentiva di poter fare, sia in base alle capacità e al tempo a disposizone.\\
Nell'ultimo caso invece è stato Dongili a staccarsi dal D5 e prendersi il compito di scrivere l'ultimo documento del progetto.\\
\begin{table}[ht]
    \centering
    \begin{tabular}{|m{0.2\linewidth}|m{0.05\linewidth}|m{0.05\linewidth}|m{0.05\linewidth}|m{0.05\linewidth}|m{0.05\linewidth}|m{0.05\linewidth}|m{0.05\linewidth}|}
    \hline
        \textbf{Nome Studente} & \textbf{D1} & \textbf{D2} & \textbf{D3} & \textbf{D4} & \textbf{D5} & \textbf{D6} & \textbf{TOT} \\
        \hline
        Lorenzo Dongili & 960 & 505 & 400 & 430 & 1395 & 360 & 4050 \\
        \hline
        Luigi Dell'Eva & 750 & 565 & 470 & 360 & 2320 & 0 & 4465 \\
        \hline
        Stefano Dal Mas & 820 & 415 & 375 & 470 & 1275 & 90 & 3445 \\
        \hline
        \textbf{Totale} & 2530 & 1485 & 1245 & 1260 & 4990 & 450 & 11960 \\
        \hline
    \end{tabular}
\end{table}