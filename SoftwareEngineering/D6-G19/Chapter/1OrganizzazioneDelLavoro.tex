\section{Organizzazione del lavoro}

Noi ragazzi del gruppo G19, non avendo alcuna esperienza nel campo, abbiamo utilizzato un approccio molto semplice.
Per i primi 4 deliverable il lavoro è stato suddiviso in piccoli task e molto spesso ci trovavamo per revisionare tutti insieme il lavoro fatto.
Per l'ultimo deliverable invece la suddivisione del lavoro è stata più netta e pesata in base all'interesse personale. Lorenzo si è concentrato nello sviluppo del FrontEnd e nella ricerca e applicazione delle api esterne (mappe ed istogrammi), Luigi ha sviluppato l'intero BackEnd, rendendo il tutto dinamico e fluido e infine Stefano si è occupato della popolazione del database, della documentazione delle api e della stesura del quinto documento.
Gli incontri tra di noi avvenivano principalmente via videochiamata ma spesso ci siamo trovati anche di persona.
Gli stumenti che sono stati utilizzati sono principalmente Discord per gli incontri, LucidChart per la realizzazione dei numerosi diagrammi, Overleaf come editor per la stesura dei documenti in latex, Photoshop per la realizzazione delle schermate nel D1 e per le immagini dell'applicazione sviluppata e infine GitHub per la fase di sviluppo.